\documentclass[]{article}
\usepackage{amssymb,amsmath}
\usepackage{ifxetex,ifluatex}
\ifxetex
  \usepackage{fontspec,xltxtra,xunicode}
  \defaultfontfeatures{Mapping=tex-text,Scale=MatchLowercase}
\else
  \ifluatex
    \usepackage{fontspec}
    \defaultfontfeatures{Mapping=tex-text,Scale=MatchLowercase}
  \else
    \usepackage[utf8]{inputenc}
  \fi
\fi
\usepackage{color}
\usepackage{fancyvrb}
\DefineShortVerb[commandchars=\\\{\}]{\|}
\DefineVerbatimEnvironment{Highlighting}{Verbatim}{commandchars=\\\{\}}
% Add ',fontsize=\small' for more characters per line
\newenvironment{Shaded}{}{}
\newcommand{\KeywordTok}[1]{\textcolor[rgb]{0.00,0.44,0.13}{\textbf{{#1}}}}
\newcommand{\DataTypeTok}[1]{\textcolor[rgb]{0.56,0.13,0.00}{{#1}}}
\newcommand{\DecValTok}[1]{\textcolor[rgb]{0.25,0.63,0.44}{{#1}}}
\newcommand{\BaseNTok}[1]{\textcolor[rgb]{0.25,0.63,0.44}{{#1}}}
\newcommand{\FloatTok}[1]{\textcolor[rgb]{0.25,0.63,0.44}{{#1}}}
\newcommand{\CharTok}[1]{\textcolor[rgb]{0.25,0.44,0.63}{{#1}}}
\newcommand{\StringTok}[1]{\textcolor[rgb]{0.25,0.44,0.63}{{#1}}}
\newcommand{\CommentTok}[1]{\textcolor[rgb]{0.38,0.63,0.69}{\textit{{#1}}}}
\newcommand{\OtherTok}[1]{\textcolor[rgb]{0.00,0.44,0.13}{{#1}}}
\newcommand{\AlertTok}[1]{\textcolor[rgb]{1.00,0.00,0.00}{\textbf{{#1}}}}
\newcommand{\FunctionTok}[1]{\textcolor[rgb]{0.02,0.16,0.49}{{#1}}}
\newcommand{\RegionMarkerTok}[1]{{#1}}
\newcommand{\ErrorTok}[1]{\textcolor[rgb]{1.00,0.00,0.00}{\textbf{{#1}}}}
\newcommand{\NormalTok}[1]{{#1}}
% Redefine labelwidth for lists; otherwise, the enumerate package will cause
% markers to extend beyond the left margin.
\makeatletter\AtBeginDocument{%
  \renewcommand{\@listi}
    {\setlength{\labelwidth}{4em}}
}\makeatother
\usepackage{enumerate}
\usepackage{graphicx}
% We will generate all images so they have a width \maxwidth. This means
% that they will get their normal width if they fit onto the page, but
% are scaled down if they would overflow the margins.
\makeatletter
\def\maxwidth{\ifdim\Gin@nat@width>\linewidth\linewidth
\else\Gin@nat@width\fi}
\makeatother
\let\Oldincludegraphics\includegraphics
\renewcommand{\includegraphics}[1]{\Oldincludegraphics[width=\maxwidth]{#1}}
\ifxetex
  \usepackage[setpagesize=false, % page size defined by xetex
              unicode=false, % unicode breaks when used with xetex
              xetex,
              colorlinks=true,
              linkcolor=blue]{hyperref}
\else
  \usepackage[unicode=true,
              colorlinks=true,
              linkcolor=blue]{hyperref}
\fi
\hypersetup{breaklinks=true, pdfborder={0 0 0}}
\setlength{\parindent}{0pt}
\setlength{\parskip}{6pt plus 2pt minus 1pt}
\setlength{\emergencystretch}{3em}  % prevent overfull lines
\setcounter{secnumdepth}{0}


\begin{document}

\section{Tutorial introductorio al curso IMSER}

Este tutorial es una forma de guiarlo a través de los conceptos y
mecanismos básicos que usan R y el curso en sí para funcionar. Se da por
sentado que usted ya vió o leyó las lecciones de la unidad 1 antes de
empezar con este texto.

A lo largo del mismo usted va a encontrar ejercicios pensados para que
usted practique y fije conceptos. Considere que este es el primer
repartido de ejercicios del curso, pero a diferencia del resto, no tiene
nota. Las soluciones están al final de este documento, puede mirarlas
cuando quiera. La idea no es que demore mucho tiempo resolviendo los
problemas, si no que entienda bien las soluciones.

\begin{quote}
Nota: en este texto usted va a encontrar fragmentos de código (comandos
o sentencias escritas en lenguaje R). Estos fragmentos son reconocibles
porque se escriben con una fuente de tipo \texttt{monospace}. Este es el
estándar en programación. Es útil porque asigna el mismo espacio a cada
caracter y por lo tanto permite ordenar fácilmente las líneas de código
(todo programador serio es muy cuidadoso en el uso de los espacios, por
diversas razones).

\end{quote}
Dado que busca abarcar muchas cosas, este tutorial es bastante extenso.
Esto es compensado en parte por ser muy sencillo de seguir. De todas
formas, el contenido está separado en cuatro secciones para dividir el
proceso en varias etapas:

\begin{enumerate}[1.]
\item
  Interfaz de R/RStudio. Se muestra cómo interaccionar con los
  principales programas usados en el curso.
\item
  Elementos básicos de la sintaxis y gramática de R. Todo lenguaje tiene
  de estos, mejor entenderlos desde el principio.
\item
  Dinámica de los repartidos. Los repartidos están diseñados para que
  usted los haga con la mayor independencia posible. Para esto usted
  debe aprender a usar el sistema de corrección incluido.
\item
  Interfaz del foro. La comunicación a través del foro es fundamental
  para plantear y resolver dudas. Vea aquí cómo funciona reddit, la
  plataforma que usaremos en el curso.
\item
  Apéndices y soluciones de ejercicios.
\end{enumerate}
\begin{center}\rule{3in}{0.4pt}\end{center}

\subsection{(1) Interfaz de R y RStudio}

La interfaz de RStudio está dividida en 4 regiones, a las que en este
curso llamaremos \emph{paneles}. En la figura 1 se muestra la
distribución de los paneles y los nombres que usaremos a lo largo del
repartido y el curso.

\begin{quote}
Nota: si es la primera vez que abre RStudio probablemente esté ausente
el panel 1 (el editor de texto plano), ya que no hay ningún archivo
abierto. Para ver dicho panel use el menú
\texttt{File \textgreater{}\textgreater{} New... \textgreater{}\textgreater{} R script}
(o la combinación de teclas: Ctrl+Shift+N).

\end{quote}
\begin{figure}[htbp]
\centering
\includegraphics{imagenes/RStudio.png}
\caption{RStudio}
\end{figure}

Al igual que en este último ejemplo, muchas veces vamos a mencionar
combinaciones de teclas, también llamados ``shortcuts'', atajos, para
ejecutar acciones en RStudio. Estos suelen ser muy prácticos. Por lo
tanto es conveniente y necesario que usted entienda la notación estándar
para estos atajos. Por ejemplo: Ctrl+Shift+K refiere a la acción de
presionar \textbf{al mismo tiempo} las teclas Ctrl (``Control''), Shift
y la letra \emph{k}. A su vez Ctrl++ implica apretar las teclas Ctrl y +
al mismo tiempo. Si usa esta combinación como ejemplo notará que RStudio
se ``recarga'' con un aumento generalizado en el tamaño de las letras.
Puede volver al tamaño original con Ctrl+-.

\subsubsection{1.1 Creación de un proyecto en RStudio}

Para organizar los archivos provistos por el curso, se creará un
directorio/carpeta en donde se guardarán estas lecciones así como el
resto de documentos y datos.

Este directorio contenerá además un proyecto de RStudio. Un proyecto de
RStudio es simplemente una forma de recordar nuestro conjunto selecto de
archivos cada vez que estamos trabajando en un tema particular. Cuando
abrimos RStudio en el proyecto ``X'', vamos a tener abiertas las mismas
pestañas que teníamos la última vez que trabajamos en X. Para crear el
directorio y el proyecto al mismo tiempo en RStudio, se puede utilizar
el menú \texttt{Project \textgreater{}\textgreater{} New Project...},
luego se elije la opción \emph{New Directory} (Fig. 2) y se crea el
directorio ``CursoR'' (aunque usted puede elegir el nombre que más le
guste). Al hacer esto el RStudio crea la carpeta y dentro de la misma un
archivo llamado \emph{CursoR.Rproj}, el cual se puede usar para abrir
RStudio con el proyecto cargado (con doble click). Además, RStudio
abrirá cargando ese proyecto por defecto cada vez, a menos que este sea
cerrado en la sesión previa
(\texttt{Project \textgreater{}\textgreater{} Close Project}).

\begin{figure}[htbp]
\centering
\includegraphics{imagenes/NewProject.png}
\caption{creación de un nuevo proyecto en RStudio}
\end{figure}

\begin{center}\rule{3in}{0.4pt}\end{center}

\paragraph{Ejercicio (0):}

como primer tarea, encárguese de crear el proyecto de RStudio para el
curso, con las instrucciones que se dieron más arriba.

\begin{center}\rule{3in}{0.4pt}\end{center}

\subsubsection{1.2 Codificación}

El término ``codificación'' refiere a la forma en que una computadora
traduce secuencias de bits, almacenados en el disco duro o la memoria,
en caracteres normales (letras, números y otros símbolos). Debido a
contingencias históricas de la informática, hay muchas formas de
establecer la codificación de un archivo. En verdad hay un subconjunto
de caracteres que son codificados de igual manera en cualquier
situación: los llamados caracteres ASCII. Es el conjunto de la mayoría
de las letras y símbolos, pero no incluye tildes o la letra eñe (en
general, nada que no se encuentre en el idioma inglés).

Debido a que los scripts están codificados a través del formato UTF-8,
lo recomendable es indicarle a RStudio que esta es la codificación que
se va a utilizar por defecto. Esta es la más común entre programadores,
independientemente del lenguaje de programación.

Utilizando la codificación correcta, se evitará encontrarse con palabras
sin sentido como ``programación'' o ``más'' en los comentarios de los
scripts (en lugar de ``programación'' o ``más''). Bajo el menú
\texttt{Tools \textgreater{}\textgreater{} Options...} (menú ``Code
Editing'') se puede modificar la codificación por defecto, como lo
indica la Figura 3.

\begin{figure}[htbp]
\centering
\includegraphics{imagenes/UTF-8.png}
\caption{Encoding}
\end{figure}

También puede ver archivos individuales con cualquier codificación a
elección, con el menú
\texttt{File \textgreater{}\textgreater{} Reopen with Encoding...}.
Tenga en cuenta que esto \emph{no equivale a cambiarle la codificación
al archivo}; para eso está la opción
\texttt{File \textgreater{}\textgreater{} Save with Encoding...}.

Una opción intermedia es configurar cada proyecto una codificación
diferente. Esto se puede hacer en el menú
\texttt{Project \textgreater{}\textgreater{} Project options...}. De
todas formas, recomendamos mantener siempre la misma codificación, en
particular, UTF-8.

\subsubsection{1.3 Directorio de trabajo}

Lo primero que debe hacer es elegir el directorio en el que va a
trabajar. Ya recomendamos crear una carpeta llamada ``CursoR'', aunque
puede llamarse como usted quiera. Si aún no la creó puede crearla
directamente con R, con el comando:

\begin{verbatim}
dir.create("CursoR")
\end{verbatim}
Escriba esto en la consola de R que se encuentra en RStudio (panel 2, de
abajo-izquierda) y presione enter. Esto va a crear una carpeta llamada
CursoR dentro del directorio de trabajo actual. Puede confirmar de dos
formas que la carpeta efectivamente fue creada: 1. con el navegador de
archivos normal de su sistema operativo o 2. con el comando
\texttt{dir}:

\begin{verbatim}
dir()
\end{verbatim}
Escribiendo esto en la consola (y dando enter) vemos una lista de
nombres de achivos y carpetas (entre comillas). Allí se debería incluir
``CursoR'' si hicimos todo bien hasta el momento.

Como recordará de la lección 1.4, usted puede saber cuál es la carpeta
de trabajo actual con la función \texttt{getwd}. De todas formas lo que
nos interesa es ``ubicarnos'' en la carpeta CursoR, así que vamos a usar
\texttt{setwd} para hacerlo:

\begin{verbatim}
setwd("CursoR")
\end{verbatim}
Nótese que aquí usamos el camino relativo para ubicarnos. Es útil de
momento, pero no siempre es lo ideal. También es posible que este
comando no sea el apropiado, si usted creó anteriormente la carpeta
``CursoR'' en un directorio de trabajo distinto al actual (en Windows,
ese directorio probablemente es la carpeta ``Mis Documentos''; en
sistemas Unix y Mac OS es \texttt{/home/user/}), o si el nombre de la
carpeta creada no es ``CursoR''. En cualquiera de estos casos R no podrá
realizar la operación y se lo hará saber con un mensaje de error impreso
en la consola (abajo-izquierda). Por ejemplo:

\begin{verbatim}
Error in setwd("CursoR") : cannot change working directory
\end{verbatim}
\begin{quote}
Nota: podríamos haber hecho lo mismo con los botones de RStudio:
\texttt{Session \textgreater{}\textgreater{} Set Working Directory \textgreater{}\textgreater{} Choose Directory}
(o el atajo Ctrl+Shift+K). Si usted usa esta alternativa, comando
\texttt{setwd("camino absoluto")} aparece igualmente en la consola y
también en la historia de comandos. Dicha historia se encuentra en el
panel de arriba-derecha, en la pestaña ``History''. Esta historia es muy
útil si queremos buscar y recuperar algún comando ejecutado hace un
tiempo (aunque no es infinita).

\end{quote}
\subsubsection{1.4 Consola vs.~editor de texto plano}

Nuestro primer comando (no escribir aún), será este:

\begin{verbatim}
mi.objeto <- 4
\end{verbatim}
Pero, ¿dónde vamos a escribirlo? Las dos opciones que veremos son
válidas, aunque no del todo iguales.

\begin{itemize}
\item
  Opción 1: la consola de RStudio. Sólo hay que dar \emph{enter} al
  terminar de escribirlo. No hace falta que el cursor esté al final de
  la sentencia para dar el enter.
\item
  Opción 2: en el editor de texto plano de RStudio (arriba-izquierda).
  Primero debe iniciar un archivo nuevo (si es que no lo hizo antes):
  vaya a
  \texttt{File \textgreater{}\textgreater{} New \textgreater{}\textgreater{} R Script}
  (Ctrl+Shift+N), recién entonces será visible este panel. En RStudio
  este panel se le llama ``Source'' (en inglés, fuente); es el ``código
  fuente'' con el que trabajamos en un momento dado. Como aún no
  guardamos el archivo, este figura bajo el nombre ``Untitled1'' (Sin
  título 1).
\end{itemize}
En general preferimos la segunda opción. Escribir los comandos en el
editor nos permite guardar todo lo que hacemos, de forma que se puede
repetir en el futuro con facilidad. Es cierto que aún si se usa la
consola los comandos se guardan en el historial (accesible en el panel
de arriba a la derecha, o con el comando \texttt{history()}), sin
embargo es fácil ver que usar el editor de texto es generalmente más
prolijo y ordenado.

De todas formas empezaremos con la opción 1, a fin de ser más
ilustrativos. A continuación, escriba nuestro comando en la consola y
presione enter:

\begin{Shaded}
\begin{Highlighting}[]
\NormalTok{mi.objeto <- }\DecValTok{4}
\end{Highlighting}
\end{Shaded}
\begin{quote}
Nota: ponga los espacios en blanco también; no afectan al comando, pero
facilitan la lectura.

\end{quote}
Debido a que usted le dió enter al comando en la consola, en su sesión
existe un objeto llamado \texttt{mi.objeto}. Puede ver una lista de los
objetos que existen en el panel de arriba a la derecha, bajo la pestaña
``Workspace'' (área de trabajo). Usaremos los términos \emph{sesión},
\emph{workspace} y \emph{área de trabajo} de forma más o menos
equivalente. También puede ver una lista de objetos existentes con el
comando \texttt{ls}: escriba en la consola

\begin{Shaded}
\begin{Highlighting}[]
\KeywordTok{ls}\NormalTok{()}
\end{Highlighting}
\end{Shaded}
Ahora veamos como usar la ``opción 2''. Como se dijo antes, preferimos
usar el editor para escribir nuestros comandos, ya que nos permite
repetirlos y organizarlos fácilmente. En RStudio se pueden ejecutar
directamente los comandos escritos en el editor de texto plano. Hay que
ubicar al cursor en la línea que nos interesa y ejecutar el atajo
Ctrl+Enter.

Escriba la línea \texttt{mi.objeto \textless{}- 4} en el editor
(arriba-izquierda) y use el atajo mencionado para que se ejecute. Si lo
hizo correctamente usted puede ver que aparece la misma línea en la
consola y se ve así:

\begin{verbatim}
> mi.objeto <- 4
>
\end{verbatim}
\subsubsection{1.5 El command prompt}

Al ver la última salida en la consola tenemos la información necesaria
para saber que el comando ya fue ejecutado, ¿cómo? gracias a la
existencia del ``command prompt'', el signo de \texttt{\textgreater{}}
que aparece al principio de cada línea en la consola.

El solitario command prompt que aparece en la última línea es un
indicador de que R ya terminó de ejecutar todo lo que se le pidió
anteriormente. Es una forma de decir ``estoy listo para recibir
órdenes''. Considere ahora las diferencias entre encontrar esto en la
consola:

\begin{verbatim}
> mi.objeto <- 4
\end{verbatim}
y esto:

\begin{verbatim}
> mi.objeto <- 4
>
\end{verbatim}
En el primer caso el comando \texttt{mi.objeto \textless{}- 4} aún no se
ejecutó (el usuario no presionó enter), mientras que en el segundo sí. A
veces el command prompt demora en aparecer, debido a que a R le toma
tiempo ejecutar el último comando. En caso de que demore demasiado,
usted puede cancelar la operación con la tecla Esc (o el botón de Stop
que tiene RStudio, en la consola).

\begin{quote}
Nota: en una terminal de Linux, en lugar de Esc se debe usar la
combinación \emph{Ctrl+C}; este es el estándar de Unix.

\end{quote}
Un error de principiante muy común es el de copiar líneas de comando
incluyendo el/los command prompt al principio. Al tratar de ejecutar
estas líneas surge un error que difícilmente pueda comprender el
usuario, ya justamete es un principiante. Es buena idea ver un ejemplo
de este error: en el editor, agregue un command propt al principio de
nuestro comando, de forma que quede así:
``\texttt{\textgreater{} mi.objeto \textless{}- 4}''. Ahora envíe esta
línea a la consola (ponga el cursor en esa línea y aprete Ctrl+Enter).
Vea el mensaje de error que devuelve R:

\begin{verbatim}
> > mi.objeto <- 4
Error: unexpected '>' in ">"
\end{verbatim}
\begin{quote}
Nota: puede que el mensaje esté en español en su PC, dependiendo del
idioma en el que haya instalado R.

\end{quote}
Borre el command prompt que acaba de agregar para evitar errores
futuros.

El command prompt es entonces una indicación útil, pero también molesta.
Muchas veces en libros o páginas web se muestran comandos de R que
empiezan con el command prompt, lo cual es desconsiderado, ya que el
usuario debe encargarse de borrar manualmente cada uno antes de poder
reproducir los ejemplos. Puede ser útil en todo caso si la idea es
diferenciar los comandos del usuario de las salidas impresas en la
consola. En este texto se usa una aproximación inversa: la salida
impresa muchas veces empieza con \texttt{\#\#} para distinguirla de los
comandos.

El command prompt tiene otra variante, el signo de \texttt{+}. El
significado es diferente, indica que los comandos anteriores no están
completos. Por ejemplo, si escribo solamente
\texttt{mi.objeto \textless{}-} va a faltar algo. Haga el ejemplo: vaya
al editor y borre el 4 al final de nuestro comando, luego envíelo a la
consola de R con el atajo Ctrl+Enter. Puede ver que en la misma aparece
lo siguiente:

\begin{verbatim}
> mi.objeto <-
+
\end{verbatim}
Esta es la forma de R de indicar que el comando no está completo. Le
está diciendo al usuario ``aún me falta algo para poder ejecutar sus
órdenes, dígame ¿qué valor debo asignar a \texttt{mi.objeto}?''. El
usuario, usted, puede completar el comando sin problemas: vaya a la
consola y escriba 4. Ahora de enter. Debería ver esto:

\begin{verbatim}
> mi.objeto <-
+ 4
>
\end{verbatim}
También tiene la opción de interrumpir el comando y volver al command
prompt normal. Alcanza con ir a la consola y apretar la tecla de escape
(Esc). En este caso se ve así en la consola:

\begin{verbatim}
> mi.objeto <-
+ 

>
\end{verbatim}
\subsubsection{1.6 Uso del autocompletar}

En la consola de R y en en algunos editores de texto (incluido el de
RStudio), es posible usar la función de autocompletar palabras. Esta se
activa antes de terminar de escribir una palabra, apretando la tecla
\emph{tab}. Siguiendo con el ejemplo anterior, si usted escribe

\begin{verbatim}
mi.o
\end{verbatim}
y apreta la tecla tab, verá como se completa el nombre
\texttt{mi.objeto}. Esto es muy útil para usar nombres largos e
informativos, con bajo riesgo de escribirlos mal. Siempre es mejor el
nombre \texttt{promedio.valores.positivos} que \texttt{p} en términos de
información dada al usuario.

\subsubsection{1.7 Mensajes de Error y de Advertencia.}

Usar R, o programar en general, es en gran medida acostumbrarse a
cometer errores. No importa el nivel de experiencia y/o habilidad del
programador, cometer errores es una constante a lo largo de la vida. La
diferencia entre un programador y un principiante no es tanto la
cantidad de errores que cometen, si no la capacidad para entenderlos y
solucionarlos. Los errores generalmente producen mensajes impresos en la
consola, los cuales suelen ser intimidantes. En realidad el mensaje de
error es una gran herramienta que usted debe aprender a usar para ser un
usuario exitoso.

En R hay dos tipos de mensajes:

\begin{enumerate}[1.]
\item
  Mensajes de \textbf{error}. Estos ocurren cuando se da un error tal
  que R no puede continuar ejecutando los comandos (con la excepción de
  que se use la función \texttt{try}, pero esto escapa a este
  repartido).
\item
  Mensajes de \textbf{advertencia} (``warnings''). Estos ocurren cuando
  se da un error que no impide continuar la ejecución de comandos.
  Debido a esto las advertencias no suelen considerarse como un asunto
  serio, pero muchas veces lo son, ya que el resultado final de la
  ejecución posiblemente esté mal.
\end{enumerate}
Puede decirse que hay otro tipo de error en R: el que no deja mensaje
alguno. Estos son los errores de lógica o escritura y que solo son
detectables siendo precavidos y haciendo pruebas para determinar si el
código hace lo que debe.

Haga un errores a propósito, es lo que hacen los profesionales. En
muchos casos, esta es una técnica muy efectiva de saber si nuestro
código funciona. Esto le beneficiará en al menos dos aspectos:

\begin{enumerate}[1.]
\item
  Aprenderá mejor cómo funcionan las cosas.
\item
  Se acostumbrará a no desesperarse cuando las cosas salen mal (i.e.:
  formará caracter).
\end{enumerate}
Veamos un ejemplo. Escriba en la consola lo siguiente:

\begin{Shaded}
\begin{Highlighting}[]
\KeywordTok{lenght}\NormalTok{(}\DecValTok{1}\NormalTok{:}\DecValTok{6}\NormalTok{)}
\end{Highlighting}
\end{Shaded}
\begin{verbatim}
## Error: could not find function "lenght"
\end{verbatim}
(Más adelante se explica el significado de \texttt{1:6}.)

Como puede ver, el comando genera un error (puede estar en español en su
PC: `Error: no se encontró la función ``lenght''\,'). El mensaje indica
que R buscó entre todas las funciones del repertorio y no encontró la
llamada ``lenght''. La enorme mayoría de las veces estos mensajes se
deben a que hay un error de escritura; en este caso la h está mal
ubicada: es \emph{length} y no lenght. Se trata de un error clásico.

A veces ocurre que R no encuentra una función u otro tipo de objeto,
como puede ser un vector o una matriz, debido a que usted no ha cargado
el paquete correcto. Por ejemplo, la función \texttt{fractions} se
encuentra en el paquete ``MASS'', el cual si bien se instala con R, no
está cargado automáticamente en su sesión de trabajo. Si escribimos el
nombre de la función y damos enter, R nos va a dar un mensaje de error:

\begin{verbatim}
> fractions
Error: object 'fractions' not found
\end{verbatim}
(Nótese que aquí se usa el término genérico ``object'', en lugar de
``function'' como en el ejemplo anterior; en aquel caso, R determinó que
lenght debió ser una función, ya que a continuación de dicha palabra
seguía la apertura de un paréntesis.)

\begin{center}\rule{3in}{0.4pt}\end{center}

\paragraph{Ejercicio (1):}

lea los mensajes de error generados con los siguientes comandos y
plantée una hipótesis de qué es lo que está mal (no intente resolverlos
si le toma mucho tiempo, mejor vea las respuestas y entienda la
solución):

\begin{Shaded}
\begin{Highlighting}[]
\KeywordTok{Mean}\NormalTok{(}\DecValTok{5}\NormalTok{:}\DecValTok{7}\NormalTok{, }\DataTypeTok{na.rm =} \OtherTok{TRUE}\NormalTok{)}
\KeywordTok{round}\NormalTok{(}\FloatTok{8.564432} \DecValTok{3}\NormalTok{)}
\KeywordTok{head}\NormalTok{(bigcity)}
\end{Highlighting}
\end{Shaded}
\begin{center}\rule{3in}{0.4pt}\end{center}

Los que hemos visto hasta aquí son errores muy simples. En la práctica,
al crear código con más sofisticación, ocurren errores mucho más
difíciles de resolver. Para esto existen técnicas, como la depuración de
código (``debugging'' en inglés), que son de enorme ayuda para
solucionarlos.

\subsection{(2) Elementos básicos de la sintaxis}

\subsubsection{2.1 Algunas manipulaciones simples}

Todas las funciones usan paréntesis luego del nombre de la mismas para
ser ejecutadas. Los usuarios más avanzados de R saben que esto no es
cierto en última instancia, pero para nosotros este será el paradigma.
Algunos ejemplos son:

\begin{verbatim}
ls()
dir()
sqrt(2)
log(16, 2)
sample(1:8, 3, replace = TRUE)
\end{verbatim}
Los paréntesis indican la región en donde el usuario ingresa los
\emph{argumentos} de la función, las entradas de la misma. Los
argumentos pueden o no ser \emph{nombrados} (el último ejemplo usa esta
opción). Por ejemplo, la función \texttt{length} sirve para saber la
cantidad de elementos de un vector cualquiera, entonces:

\begin{Shaded}
\begin{Highlighting}[]
\NormalTok{x <- }\DecValTok{3}\NormalTok{:}\DecValTok{6}
\KeywordTok{length}\NormalTok{(x)}
\end{Highlighting}
\end{Shaded}
\begin{quote}
Nota: el vector \texttt{x} es la secuencia de números enteros 3, 4, 5, y
6.

\end{quote}
Escriba este ejemplo. Luego de ejecutar el comando, en la consola
debería mostrar impresa la salida de \texttt{length(x)}, que es el valor
4. Como dijimos, los paréntesis delimitan el conjunto de lo que son
``las entradas'' de una función. Aquí hay una sola entrada: el vector
\texttt{x}. Hay varios lenguajes de programación que no usan este
esquema, pero no son la mayoría. Por otro lado, ya vimos que la salida
aquí es 4 y que luego de ser impresa en la consola ``se pierde''. Como
usuarios podemos guardar este valor en un objeto, tal como se mostrara
en la lección 1.2. Por ejemplo:

\begin{Shaded}
\begin{Highlighting}[]
\NormalTok{y <- }\KeywordTok{length}\NormalTok{(x)}
\end{Highlighting}
\end{Shaded}
Este comando guarda la salida de \texttt{length(x)} en un nuevo objeto,
\texttt{y}. La ``flecha'' hacia la izquierda, \texttt{\textless{}-} es
el operador que normalmente se usa para hacer asignaciones (hay al menos
seis formas de hacer asignaciones, pero en general sólo usaremos la
flecha a la izquierda).

\begin{center}\rule{3in}{0.4pt}\end{center}

\paragraph{Ejercicio (2):}

haga usted un ensayo con la función \texttt{mean} (para calcular
promedios). La entrada será otra vez el vector \texttt{x} y la salida un
objeto llamado \texttt{promedio}.

\begin{center}\rule{3in}{0.4pt}\end{center}

Si usted hizo todo bien, entonces en el panel 3 (arriba-derecha) de
RStudio, bajo la pesataña \emph{Workspace}, encontrará al objeto
\texttt{promedio} en la lista de objetos presentes y su valor será 2.5.
También puede ejecutar:

\begin{Shaded}
\begin{Highlighting}[]
\KeywordTok{exists}\NormalTok{(}\StringTok{"promedio"}\NormalTok{)}
\end{Highlighting}
\end{Shaded}
(Si el resultado es \texttt{TRUE} entonces \texttt{promedio}
``existe''.)

En todo momento puede usar la consola para inspeccionar el objeto, con
sólo escribir el nombre (escriba y de enter):

\begin{Shaded}
\begin{Highlighting}[]
\NormalTok{promedio}
\end{Highlighting}
\end{Shaded}
\subsubsection{2.2 Secuencias de números}

Como habrá notado al crear \texttt{x}, el símbolo \texttt{:} sirve para
crear secuencias de números enteros. Por ejemplo \texttt{0:4} son los
números 0, 1, 2, 3 y 4.

\begin{center}\rule{3in}{0.4pt}\end{center}

\paragraph{Ejercicio (3):}

usando \texttt{:} genere las siguientes secuencias de números enteros:

\begin{enumerate}[1.]
\item
  Los números de 10 al 10000.
\item
  Los números del 20 al 10 (sí, es en orden decreciente).
\item
  Los números del -8 al 6 en orden creciente.
\item
  Los números del -8 al 6 en orden decreciente.
\end{enumerate}
\begin{center}\rule{3in}{0.4pt}\end{center}

Pero además del operador \texttt{:}, existe la función \texttt{seq}, que
sirve cuando la secuencia (regular) no es de números \emph{enteros}, o
la distancia entre consecutivos es distinta de 1. Por ejemplo, la
secuencia 0, 0.2, 0.4, 0.6, 0.8 y 1 se puede crear así:

\begin{Shaded}
\begin{Highlighting}[]
\KeywordTok{seq}\NormalTok{(}\DecValTok{0}\NormalTok{, }\DecValTok{1}\NormalTok{, }\DataTypeTok{by =} \FloatTok{0.2}\NormalTok{)}
\end{Highlighting}
\end{Shaded}
\begin{verbatim}
## [1] 0.0 0.2 0.4 0.6 0.8 1.0
\end{verbatim}
Nótese que entre paréntesis está: el inicio, el final y la distancia
entre los valores consecutivos (indicado por el nombre de argumento:
``\texttt{by}''). Es importante destacar que los tres argumentos están
separados por \emph{comas}. Alternativamente, se puede crear una
secuencia indicando el inicio, el final y la cantidad de valores del
vector de salida:

\begin{Shaded}
\begin{Highlighting}[]
\KeywordTok{seq}\NormalTok{(}\DecValTok{0}\NormalTok{, }\DecValTok{1}\NormalTok{, }\DataTypeTok{length =} \DecValTok{11}\NormalTok{)}
\end{Highlighting}
\end{Shaded}
\begin{verbatim}
##  [1] 0.0 0.1 0.2 0.3 0.4 0.5 0.6 0.7 0.8 0.9 1.0
\end{verbatim}
\begin{center}\rule{3in}{0.4pt}\end{center}

\paragraph{Ejercicio (4):}

usando \texttt{seq} genere las siguientes secuencias de números:

\begin{enumerate}[1.]
\item
  Los números pares del 2 al 110.
\item
  Los números impares del 1 al 110.
\item
  Un vector de 101 elementos, con valores desde 9 hasta 0 (orden
  decreciente).
\end{enumerate}
\begin{center}\rule{3in}{0.4pt}\end{center}

\subsubsection{2.3 Pegando cosas}

Otra función sumamente útil es la de concatenación: \texttt{c}. Sirve
para ``armar'' o ``pegar'' elementos y así crear un vector. Ejecute el
siguiente ejemplo:

\begin{Shaded}
\begin{Highlighting}[]
\NormalTok{mi.vector <- }\KeywordTok{c}\NormalTok{(x, promedio, }\DecValTok{14}\NormalTok{)}
\end{Highlighting}
\end{Shaded}
El resultado es un nuevo vector llamado \texttt{mi.vector}, con 6
elementos (vea la salida de \texttt{length(mi.vector)} para
confirmarlo).

\begin{center}\rule{3in}{0.4pt}\end{center}

\paragraph{Ejercicio (5):}

escriba ahora el comando necesario para crear un vector llamado
\texttt{mi.otro.vector}, el cual tendrá la secuencia de valores 45, -76,
3, 4, 5, 6, 0.333.

\begin{center}\rule{3in}{0.4pt}\end{center}

\begin{quote}
Nota: en R hay tres funciones de concatenación: \texttt{c},
\texttt{rbind} y \texttt{cbind}. Acabamos de ver la primera, las otras
dos servián cuando trabajemos con matrices o data.frames.

\end{quote}
\subsubsection{2.4 Cambiando valores}

Hasta ahora hemos visto como crear vectores y hacer asignaciones. Veamos
ahora la forma de modificarlos. Supongamos que queremos cambiar el
último valor de \texttt{mi.vector}; el nuevo valor será 0. Una forma
fácil de hacerlo es así:

\begin{Shaded}
\begin{Highlighting}[]
\NormalTok{mi.vector[}\DecValTok{6}\NormalTok{] <- }\DecValTok{0}
\end{Highlighting}
\end{Shaded}
Aquí estamos utilizando los corchetes o paréntesis rectos para
\emph{modificar} un vector. En verdad no es más que un caso particular
de la operación de asignación. Los paréntesis rectos sirven para indicar
la ubicación de \emph{\textbf{el o los}} elemento(s) que quiero cambiar.

Se podrían modificar \emph{varios} elementos de \texttt{mi.vector} al
mismo tiempo también. En lugar de poner un único valor entre corchetes,
se puede poner un \emph{vector} con las posiciones que quiere modificar.
Por ejemplo, cambiar los valores \texttt{mi.vector{[}1{]}} y
\texttt{mi.vector{[}4{]}} por -1 se puede hacer con el comando:

\begin{Shaded}
\begin{Highlighting}[]
\NormalTok{mi.vector[}\KeywordTok{c}\NormalTok{(}\DecValTok{1}\NormalTok{, }\DecValTok{4}\NormalTok{)] <- -}\DecValTok{1}
\end{Highlighting}
\end{Shaded}
Nótese que se usa la concatenación \texttt{c} para primero formar el
vector y luego ponerlo entre los corchetes. Si en cambio escribiéramos
\texttt{mi.vector{[}1, 4{]}} estaríamos cometiendo un error (que usted
entenderá cuando aprenda más sobre matrices y data.frames).

\begin{center}\rule{3in}{0.4pt}\end{center}

\paragraph{Ejercicio (6):}

modifique las posiciones 2 y 3 de \texttt{mi.vector}, sustituyéndolos
por 100 y 104 respectivamente \emph{en un sólo paso} (hacerlo en 2 pasos
es trivial y para nada elegante). Esta es una situación diferente a los
ejemplos anteriores, ya que hay 2 valores reemplazantes. Para hacerlo
correctamente tendrá que usar la función \texttt{c} del lado derecho de
la ``flecha'' de asignación.

\begin{center}\rule{3in}{0.4pt}\end{center}

\subsubsection{2.5 Extrayendo valores}

Los corchetes también sirven para extraer valores. Ejecute el siguiente
comando a modo de ejemplo:

\begin{Shaded}
\begin{Highlighting}[]
\NormalTok{mi.vector[}\KeywordTok{c}\NormalTok{(}\DecValTok{2}\NormalTok{, }\DecValTok{5}\NormalTok{)]}
\end{Highlighting}
\end{Shaded}
En la consola debería ver los valores 4.0 y 4.5. Al igual que antes,
esta salida se puede guardar en un objeto nuevo:

\begin{Shaded}
\begin{Highlighting}[]
\NormalTok{u <- mi.vector[}\KeywordTok{c}\NormalTok{(}\DecValTok{3}\NormalTok{, }\DecValTok{4}\NormalTok{, }\DecValTok{5}\NormalTok{)]}
\CommentTok{# O también}
\NormalTok{u <- mi.vector[}\DecValTok{3}\NormalTok{:}\DecValTok{5}\NormalTok{]}
\end{Highlighting}
\end{Shaded}
\begin{quote}
Nota: muchas veces quienes aprenden R escriben expresiones como
\texttt{c(3:5)}, lo cual es un uso innecesario de \texttt{c}. Alcanza
con poner \texttt{3:5} para hacer lo mismo.

\end{quote}
Este método de extraer valores puede ser muy útil para reordenar un
vector. Por ejemplo el siguiente comando:

\begin{Shaded}
\begin{Highlighting}[]
\NormalTok{mi.vector[}\KeywordTok{c}\NormalTok{(}\DecValTok{4}\NormalTok{:}\DecValTok{6}\NormalTok{, }\DecValTok{1}\NormalTok{:}\DecValTok{3}\NormalTok{)]}
\end{Highlighting}
\end{Shaded}
\begin{verbatim}
## [1] -1.0000  0.7989  0.0000 -1.0000 -0.4869 -0.2107
\end{verbatim}
Cambia de lugar las dos mitades del vector original.

\begin{center}\rule{3in}{0.4pt}\end{center}

\paragraph{Ejercicio (7):}

busque usted la forma de obtener un vector con los valores de
\texttt{mi.vector} pero en orden inverso.

\begin{center}\rule{3in}{0.4pt}\end{center}

\begin{quote}
Nota: la extracción o modificación de valores funciona de forma similar
con ciertos tipos de objetos: factores, matrices, data.frames o listas.
Los detalles vendrán en otras unidades del curso.

\end{quote}
\subsubsection{2.6 Funciones \& operadores}

Casi todos los comandos que se ejecutan en la práctica involucran
objetos que se califican como \emph{funciones} u \emph{operadores}. La
línea que separa entre uno y otro es arbitraria pero útil en la
práctica, de la misma manera en que es útil en la cocina diferenciar
frutas de verduras. Para los objetivos de este curso, alcanza con decir
que:

\begin{enumerate}[1.]
\item
  Las funciones son todas aquellas que tienen un nombre escrito (i.e.:
  una ``palabra'') y que para ejecutarse requieren de paréntesis, a fin
  de indicar las ``entradas'' (los valores de los argumentos). Ejemplos
  típicos son \texttt{mean}, \texttt{length}, \texttt{print},
  \texttt{read.table}, \texttt{c}, \texttt{setwd} y \texttt{ls}, así
  como las funciones creadas por el propio usuario.
\item
  Los operadores son aquellos que escribimos con uno o pocos símbolos
  (típicamente 1 o 2) y que sirven para realizar las operaciones más
  comunes. Ejemplos típicos son \texttt{+}, \texttt{-}, \texttt{*},
  \texttt{/}, \texttt{**}, \texttt{\^{}}, \texttt{:},
  \texttt{\textless{}-}, \texttt{{[}}, \texttt{\%*\%}, \texttt{\$} y
  \texttt{==}.
\item
  Hay algunos objetos que no son funciones ni operadores. Les llamaremos
  ``constructos'' generalmente. Afortunadamente no son muchos y todos se
  encuentran en lo que llamaremos \emph{Estructuras de control}. Estos
  son: \texttt{for}, \texttt{while}, \texttt{repeat}, \texttt{break},
  \texttt{next}, \texttt{if} y \texttt{else} (puede consultar la ayuda
  de R con \texttt{?Control}).
\end{enumerate}
En general asociamos a los operadores con operaciones matemáticas (suma,
resta, multiplicación), mientras que las funciones pueden asociarse con
``programas'', ya que pueden realizar procedimientos más complejos. Sin
embargo esta división es artificial, ya que muchas operaciones
matemáticas se deben realizar con funciones (según la definición
anterior). Algunos ejemplos son: \texttt{sin} (función seno),
\texttt{sqrt} (raíz cuadrada), \texttt{sum} (sumatoria), \texttt{mean}
(media aritmética) e \texttt{integrate} (integración numérica).

\paragraph{Orden de precedencia.}

En el caso de los operadores, existe una convención respecto al orden
temporal en que deben evaluarse. Esto es necesario para evitar
ambigüedades. Por ejemplo:

\begin{verbatim}
4 + 2 * 9
\end{verbatim}
¿Debo sumar 4 al producto de $2 \cdot 9$ o multiplicar por 9 la adición
de $4 + 2$? Los resultados serán distintos según cuál es el operador que
se ejecute primero. Por convención es la multiplicación la que se
ejecutará primero (puede hacer la prueba en R). En caso de duda, siempre
puede usar paréntesis para asegurarse una ejecución correcta; ej.:
\texttt{4 + (2 * 9)}. En este caso es bastante intuitiva la solución a
la pregunta, pero en otros no tanto. Para esto tenemos una tabla que
puede consultar cuando quiera, que indica el \emph{orden de precedencia}
de los operadores. Puede seguir
\href{https://www.dropbox.com/s/7a3q3xgax777zdq/2.0.1-orden-de-precedencia.pdf}{este
link} para ver dicha tabla.

\begin{center}\rule{3in}{0.4pt}\end{center}

\paragraph{Ejercicio (8):}

su tarea es encontrar las soluciones numéricas a los problemas que se
plantean, traduciendo correctamente las expresiones matemáticas a
expresiones en lenguaje R. Es posible que usted necesite consultar la
ayuda para hacer el ejercicio. Si no sabe exactamente cuál es el nombre
de la función que quiere consultar, use \texttt{help.search}, con
cualquiera de sus dos modalidades (ver lección 1.3). Por ejemplo, si
quiere averiguar sobre integrales, puede ejecutar:

\begin{verbatim}
help.search("integral")
??integral
\end{verbatim}
Si en cambio usted ya sabe bien a qué página quiere ir, como por ejemplo
la ayuda de \texttt{integrate}, puede usar la función \texttt{help}, que
también tiene dos modalidades:

\begin{verbatim}
help("integrate")
?integrate
\end{verbatim}
Estos ejercicios son simples pero con dificultad muy variada, no se
espera que usted los pueda hacer todos la primera vez. Recuerde que las
respuestas están a la vista; no es la idea que usted se detenga
demasiado tiempo en estos ejercicios.

De todas formas, \textbf{es importante} que usted entienda las
soluciones. Si tiene alguna pregunta relevante, no dude en consultar el
foro del curso (ver la sección 4 de este tutorial). Puede que la
pregunta ya se haya hecho, o puede ser usted quien la formule.

\begin{quote}
Nota: considere usar un traductor en línea (como \emph{Google
translate}) para leer la documentación, si representa un obstáculo el
idioma.

\end{quote}
\begin{enumerate}[a.]
\item
  Encuentre la(s) función necesaria para calcular logaritmos. Vaya a la
  página de ayuda de la función correspondiente y encuentre cuál es el
  nombre del argumento usado para determinar la base del logaritmo.
  Luego calcule los siguientes logaritmos: $log_3 8$, $log_5 13$,
  $log_{10} 1.5$, $ln 7$, $log_{2} 6$.
\item
  Encuentre la función que devuelve los extremos (máximo y mínimo) de un
  vector, el \emph{rango} de valores. Vaya a la página de ayuda de la
  misma. Debe encontrar un argumento que sirve para cambiar el
  comportamiento de la función, si el vector de entrada contiene valores
  no finitos (\texttt{Inf} o \texttt{-Inf}). Si dicho argumento está
  ``activado'', los valores infinitos se omiten. Determine el rango de
  los vectores (descartando infinitos):

\begin{verbatim}
x <- 1:100
x <- -100:-1
x <- c(pi, runif(10, 4, 6), exp(8), Inf)
\end{verbatim}
\item
  Calcule los valores que toma el siguiente polinomio para los valores
  de $x$: -1, 1.5 y 4
\end{enumerate}
\[
  5 x ^ 3 - 2 x ^ 2 + 11
\]

\begin{enumerate}[a.]
\setcounter{enumi}{3}
\item
  ¿Cuál es el valor de $h$ para cualquier valor de $x$ entre -1 y 1?
\end{enumerate}
\[
  h = cos(x) ^ 2 + sen(x) ^ 2
\]

\begin{enumerate}[a.]
\setcounter{enumi}{4}
\item
  Calcule los valores de $D$ para $a = 1, 2, 3, 4, ..., 100$:
\end{enumerate}
\[
  D = a \cdot \sqrt{2}
\]

\begin{enumerate}[a.]
\setcounter{enumi}{5}
\item
  Calcule el valor de la siguiente sumatoria para los valores de
  $n = 1, 5, 10$
\end{enumerate}
\[
  \sum_{i = 0} ^{i = n} \frac{1}{i!}
\]

\begin{enumerate}[a.]
\setcounter{enumi}{6}
\item
  Calcule el valor de la siguiente integral:
\end{enumerate}
\[
  \int_{0} ^{20} ln(x) \; dx
\]

\begin{center}\rule{3in}{0.4pt}\end{center}

\subsection{(3) Repartidos}

Los repartidos del curso tienen un sistema de corrección ``automática''
que usted debe aprender a usar. El objetivo de esto es lograr que
estudiante no dependa de la supervisión de los profesores para saber si
sus ejercicios están correctos. Este sistema consta de dos tipos de
archivos: los que debe editar (ejercicios) y los que no. Los archivos se
encuentrarán en la carpeta del repartido (ver más abajo).

En general se plantean problemas para los que usted tiene que crear o
modificar el código necesario para resolverlo. Siempre que es
pertinente, se piden soluciones genéricas a los problemas, es decir, que
sirven para una amplia variedad de casos en lugar de uno en particular.

Al tratarse de un curso completamente en línea es imposible controlar
por completo el copiado de soluciones. Por esta razón todos los
estudiantes deben aceptar el \textbf{código de honor} del curso. El
mismo se encuentra en el \textbf{Apéndice I}, al final de este
documento.

A continuación se explica con ejemplos la dinámica de los repartidos.

\subsubsection{3.1 Soluciones ``genéricas''}

\begin{quote}
(Nota: si el vector \texttt{mi.vector} no está en su sesión de trabajo
en este momento, repita los pasos para volver a crearlo y recuerde que
debe tener 6 elementos; alternativamente, construya un nuevo
\texttt{mi.vector} cualquiera con 6 valores.)

\end{quote}
Anteriormente quisimos cambiar el último valor de \texttt{mi.vector}.
Para eso corrimos este comando:

\begin{Shaded}
\begin{Highlighting}[]
\NormalTok{mi.vector[}\DecValTok{6}\NormalTok{] <- }\DecValTok{0}
\end{Highlighting}
\end{Shaded}
Llamémosle a esta la ``solución 1'' de nuestro problema. Esta solución
es perfecta si \texttt{mi.vector} tiene 6 elementos. ¿Qué pasa si no
sabemos de antemano esta información?

Ya vimos que la función \texttt{length} sirve justamente para
averiguarlo. Por lo tanto debe haber una manera de usarla para no
depender de ese conocimiento. ¿Puede imaginar esta alternativa? Lo
invito a que lo intente. De todas formas, una solución es la siguiente:

\begin{Shaded}
\begin{Highlighting}[]
\NormalTok{mi.vector[}\KeywordTok{length}\NormalTok{(mi.vector)] <- }\DecValTok{0}
\end{Highlighting}
\end{Shaded}
Es importante que entienda lo que se hizo aquí.

El comando \texttt{length(mi.vector)} devuelve la longitud de
\texttt{mi.vector} (i.e.: la ubicación que queremos modificar); ese
valor es enviado a ``los corchetes'' para definir correctamente el valor
que hay que modificar. Esto es lo que podríamos llamar una
\emph{composición de funciones} (para quienes quieran profundizar, las
dos funciones involucradas son \texttt{length} y
\texttt{{[}\textless{}-}; la página de ayuda de la segunda es accesible
con el comando \texttt{?"{[}\textless{}-"}).

Podríamos extendernos y encontrar una ``solución 3'' si dividimos la
tarea en dos pasos:

\begin{Shaded}
\begin{Highlighting}[]
\NormalTok{ubicacion <- }\KeywordTok{length}\NormalTok{(mi.vector)}
\NormalTok{mi.vector[ubicacion] <- }\DecValTok{0}
\end{Highlighting}
\end{Shaded}
Esta solución tiene sentido si vamos a volver a utilizar el objeto
\texttt{ubicacion} en el futuro. Separar los pasos también puede lograr
que una secuencia de comandos sea más fácil de leer; en el curso solemos
hacer este tipo de separaciones, pero es una estrategia que responde a
razones pedagógicas y no a que sea necesariamente la mejor opción del
punto de vista práctico.

Lo importante a destacar de las soluciones 2 y 3 es que son
\emph{genéricas}: no importa qué tan grande o chico sea el vector
\texttt{mi.vector}, siempre será una solución correcta. Lo opuesto a una
solución genérica es una solución \emph{específica}, como la solución 1.

La utilidad de la programación reside en gran medida en encontrar
soluciones genéricas a problemas prácticos, de forma que un mismo código
pueda ser aplicable a una variedad de situaciones. Por esto el énfasis
en el curso es el de encontrar siempre la solución más genérica posible.

\begin{center}\rule{3in}{0.4pt}\end{center}

\paragraph{Ejercicio (9):}

anteriormente se le pidió el código necesario para obtener los valores
de \texttt{mi.vector} pero en orden invertido. En aquella ocasión la
solución específica era suficiente. Ahora debe encontrar una solución
\emph{genérica} para el mismo problema. En otras palabras, considere que
\texttt{mi.vector} puede tener una longitud cualquiera mayor o igual a
dos y que usted desconoce.

\begin{center}\rule{3in}{0.4pt}\end{center}

Durante todo el curso los ejercicios sólo se considerarán correctos si
resuelven el problema de forma genérica. Por lo tanto, ante un ejercicio
como el anterior, la respuesta \texttt{mi.vector{[}6:1{]}} será
incorrecta; la respuesta genérica
\texttt{mi.vector{[}length(mi.vector):1{]}} será apropiada (pero no
necesariamente la \emph{única forma} de obtener una respuesta correcta).

\begin{quote}
Nota: en R la función \texttt{rev} sirve para ``dar vuelta'' vectores
(ej.: \texttt{rev(1:10)}).

\end{quote}
\subsubsection{3.2 El formato de los ejercicios}

Cada repartido se compone de un archivo zip que usted debe descomprimir
dentro de la carpeta del curso. Para continuar con el Repartido I,
descargue de la web el archivo \href{http://goo.gl/2evbBS}{rep-1.zip}.
Debe extraer sus contenidos en la carpeta del curso (``CursoR'' o el
nombre que usted le haya dado), de forma que tendrá una subcarpeta
llamada ``rep-1''. Allí encontrará una serie de archivos. En algunos
usted deberá escribir código R para solucionar algún problema, estos son
los archivos de los ejercicios. Los otros debe dejarlos como están, sin
moverlos ni cambiar su nombre.

Para continuar con los ejemplos, es necesario que usted ya tenga lista
la carpeta del repartido, haya abierto una sesión de R y que dicha
carpeta sea el directorio de trabajo actual. Puede usar \texttt{getwd()}
para comprobar que está trabajando en la carpeta correcta, como se
indica en la lección 1.4.

El siguiente paso es abrir en el editor de RStudio el archivo del primer
ejercicio, ``1-varianza.R'' (puede usar el atajo Ctrl+O).

Antes de continuar, lea las instrucciones del primer ejercicio,
contenidas en el archivo letra.pdf. No se preocupe si no las entiende
del todo, iremos paso a paso para resolverlo. Luego vuelva a RStudio y
al archivo 1-varianza.R.

Allí encontrará que hay dos tipos de líneas: las que tienen código
funcional (``que hace cosas'') y líneas con instrucciones que empiezan
con \texttt{\#}. En una \emph{línea} de código R, todo lo que se escribe
después del signo \texttt{\#} es \emph{comentario} y no es
evaluado/ejecutado (la idea es dejar mensajes al próximo usuario, muchas
veces el propio autor, para que pueda entender mejor lo que hace el
código). Lea con tranquilidad las instrucciones de este archivo antes de
continuar.

En cada archivo de ejercicio hay unas líneas dedicadas a los objetivos
del mismo. Estos generalmente son objetos que su código debe producir.
En este caso hay tres: \texttt{x\_mean}, \texttt{s} y \texttt{out}. Note
que en las descripciones se pone un asterisco a aquellos objetivos que
son obligatorios para completar el ejercicio. Si no tiene un asterisco,
debe tomarlo como una sugerencia para resolver más fácilmente el
problema.

Siguiendo con las líneas del archivo, puede ver también que la región en
la que usted debe trabajar está claramente delimitada:

\begin{Shaded}
\begin{Highlighting}[]
\CommentTok{# ===== Su código comienza aquí: =====#}

\NormalTok{x_mean <- }\DecValTok{0}

\NormalTok{s <- }\DecValTok{0}

\NormalTok{out <- }\DecValTok{0}

\CommentTok{# ====== Aquí finaliza su código =====#}
\end{Highlighting}
\end{Shaded}
Es decir, usted puede escribir lo que quiera entre medio de las dos
líneas de comentario copiadas aquí. También puede aumentar o disminuir
la cantidad de líneas que hay entre medio, sin límites.

Puede escribir código por fuera de este espacio, pero tenga en cuenta
que:

\begin{enumerate}[1.]
\item
  Esas líneas van a ser descartadas en el proceso de corrección.
\item
  No es buena idea modificar el código que ya viene en el archivo
  original.
\end{enumerate}
\subsubsection{3.3 Usando la función evaluar}

Ahora usted empezará a editar el archivo 1-varianza.R para ir logrando
los objetivos que pide el ejercicio.

Nótese que en la línea 2 hay un comando necesario para crear un vector
\texttt{x} aleatorio. Este comando se puede correr individualmente (por
ejemplo, con el atajo Ctrl+Enter) o junto con todo el archivo:

\begin{Shaded}
\begin{Highlighting}[]
\KeywordTok{source}\NormalTok{(}\StringTok{"1-varianza.R"}\NormalTok{)}
\end{Highlighting}
\end{Shaded}
(O en RStudio: Ctrl+Shift+S; note como al hacer este atajo aparece en la
consola el comando correspondiente con el camino absoluto al archivo.)

Una vez que tenemos creado \texttt{x} podemos empezar a jugar con el
archivo para buscar las soluciones correctas al problema.

Lo primero que resolveremos será el objeto \texttt{x\_mean}. Tenemos que
modificar el código para que dicho objeto sea la media aritmética de el
vector \texttt{x}. En las instrucciones del script se indica que usted
debe usar la ayuda de R para encontrar la función que necesita; esto
será común a lo largo del curso. De todas formas ya hemos mencionado que
la función \texttt{mean} es la que precisamos para la tarea. Lo que debe
hacer entonces es modificar la línea

\begin{Shaded}
\begin{Highlighting}[]
\NormalTok{x_mean <- }\DecValTok{0}
\end{Highlighting}
\end{Shaded}
sustituyendo el 0 por \texttt{mean(x)}:

\begin{Shaded}
\begin{Highlighting}[]
\NormalTok{x_mean <- }\KeywordTok{mean}\NormalTok{(x)}
\end{Highlighting}
\end{Shaded}
Ahora bien, como \texttt{x\_mean} es uno de los objetivos del ejercicio,
podemos verificar si lo hemos hecho bien o no. Para eso se necesita la
función \texttt{evaluar}, que es el eje central del mecanismo de
corrección. Lo primero es guardar los cambios en el archivo (Ctrl+S).

Ahora corra el archivo evaluar.R con el comando:

\begin{Shaded}
\begin{Highlighting}[]
\KeywordTok{source}\NormalTok{(}\StringTok{"evaluar.R"}\NormalTok{, }\DataTypeTok{encoding =} \StringTok{"UTF-8"}\NormalTok{)}
\end{Highlighting}
\end{Shaded}
\begin{quote}
(Nota: si usted trabaja en Mac OS o Linux, puede omitir el argumento
\texttt{encoding = "UTF-8"}.)

\end{quote}
Cada vez que cargamos este archivo se imprime en la consola un mensaje:

\begin{verbatim}
Funciones cargadas correctamente: evaluar, verNotas y fecha.datos

Chequeo de encoding:
  Los siguientes caracteres deben ser vocales con tilde:
  á - é - í - ó - ú
  Si *no se ven correctamente* corra el siguiente comando:
  source('evaluar.R', encoding = 'UTF-8')

Para comprobar la fecha de su archivo datos ejecute:
>> fecha.datos()
\end{verbatim}
La primer parte del mensaje se explica sola, lo último, referente al
``archivo datos'', lo explicaremos más adelante.

El resultado que nos importa es que ahora está cargada la función
\texttt{evaluar}, necesaria para corregir el ejercicio. Para usarla,
basta con correr:

\begin{Shaded}
\begin{Highlighting}[]
\KeywordTok{evaluar}\NormalTok{()}
\end{Highlighting}
\end{Shaded}
Tras esta acción se imprimirá en la consola un menú para elegir cuál
ejercicio desea corregir. En este caso será el 1:

\begin{verbatim}
Elija el archivo que desea corregir: 

1: Ej. (1): 1-varianza.R
2: Ej. (2): 2-zenon.R
3: Ej. (3): 3-extra-dist.R
4: Todos

Selection: 1
\end{verbatim}
(Alternativamente puede escribir \texttt{evaluar(1)} directamente y así
saltear el menú.)

Al hacer esto se imprime una cantidad considerable de texto en la
consola:

\begin{verbatim}
>> Iniciando una nueva semilla:
>> set.seed(444)
>> Creando un nuevo vector x para la corrección:
>> x <- rnorm(sample(100, 200, 1))
>> el nuevo x tiene 20 elementos; su varianza es 0.9023475
valor de x_mean ... OK

==========RESULTADOS==========

El script "1-varianza.R" tiene algún error, lo siento :-(
    -> Verifique que su solución sea genérica y que sigue
       todas las consignas de la letra. 

Ejercicios correctos:

==>> Ninguno por ahora

Se generaron los siguientes mensajes de error:

* Al corregir el ej. 1, archivo 1-varianza.R:
Error : la longitud del vector s no es la esperada,
la diferencia observada es la siguiente:

  Obs. Esp. Dif. 
     1   20   19 

==============================

Total hasta ahora: 0 de 1 ejercicios; NOTA: 0 % 
\end{verbatim}
\textbf{Empecemos por las cinco primeras líneas:}

\begin{verbatim}
>> Iniciando una nueva semilla:
>> set.seed(444)
>> Creando un nuevo vector x para la corrección:
>> x <- rnorm(sample(10, 20, 1))
>> el nuevo x tiene 20 elementos; su varianza es 0.9023475
\end{verbatim}
Esta información es necesaria si desea reproducir el ejemplo usado en la
corrección. En primer lugar se indica que se usa una ``semilla'', con el
comando \texttt{set.seed(444)}. Esto sirve para poder reproducir el
ejemplo exactamente y es necesario, ya que se usan generadores de
números aleatorios como \texttt{rnorm} y \texttt{sample}. Luego se
muestra el comando usado para generar el nuevo \texttt{x}. En
definitiva, usted puede reproducir exactamente los primeros pasos del
mecanismo de corrección con los comandos:

\begin{Shaded}
\begin{Highlighting}[]
\KeywordTok{set.seed}\NormalTok{(}\DecValTok{444}\NormalTok{)}
\NormalTok{x <- }\KeywordTok{rnorm}\NormalTok{(}\KeywordTok{sample}\NormalTok{(}\DecValTok{10}\NormalTok{, }\DecValTok{20}\NormalTok{, }\DecValTok{1}\NormalTok{))}
\end{Highlighting}
\end{Shaded}
La preparación culmina con una indicación de las propiedades básicas de
\texttt{x}. Note que estas primeras líneas comienzan con
\texttt{\textgreater{}\textgreater{}}, para evitar confusiones con
comandos que haya ejecutado usted anteriormente.

En la sexta línea (``\texttt{valor de x\_mean ... OK}'') se muestra que
nuestro objeto \texttt{x\_mean} parece estar correcto. Muchas veces las
correcciones tienen varios pasos intermedios antes de completarse. Por
cada paso superado se imprime una línea como esta, con un
``\texttt{... OK}'' al final. Esté atento a este detalle, ya que puede
ser de gran ayuda para entender qué parte del problema no está
resolviendo bien.

A continuación se abre un espacio con resultados. Aquí lo más importante
son los mensajes de error. Los errores se indican empezando con un
asterisco, el número de ejercicio y el archivo en el que se detectaron.
Debido a que los errores detienen la ejecución de R, nunca se muestra
más de un error \emph{por archivo}.

En este caso el mensaje reza ``la longitud del vector s no es la
esperada'', lo cual es obvio para nosotros, ya que en el archivo sigue
estando la línea \texttt{s \textless{}- 0} (a propósito, el 0 se pone
inicialmente para poder correr todo el script sin que ocurran errores,
aún si usted no empezó con el ejercicio).

Pero el mensaje de error busca dar un poco más de información, indicando
cuál era el valor esperado y comparándolo con lo que se observó. Mire de
vuelta el mensaje de error y fíjese cuál es el valor observado, cuál el
esperado y cuánto es la diferencia. ¿Tiene sentido?. Esta información,
junto con la que se da en las líneas con
\texttt{\textgreater{}\textgreater{}} al principio, es muy útil para
determinar en qué parte del ejercicio está el problema.

Por último el mensaje le muestra la nota acumulada hasta el momento. Más
adelante iremos más en detalle con este aspecto.

Vayamos entonces al problema del objeto \texttt{s}. En las instrucciones
del archivo 1-varianza.R se indica claramente que 1) \texttt{s} es un
vector y que 2) es la entrada de la función \texttt{sum}.
Específicamente, el vector \texttt{s} debe tener todos los términos de
la sumatoria necesaria para calcular la varianza de \texttt{x}. Sabemos
que la fórmula de la varianza es:

\[
  \sigma ^ 2 = \frac{1}{n - 1} \cdot \sum_{i=1}^{i=n} (x_i - \overline{x}) ^ 2 
\]

Por lo tanto el iésimo elemento de \texttt{s} debe ser
$(x_i - \overline{x}) ^ 2$, lo cual se traduce al siguiente código:

\begin{Shaded}
\begin{Highlighting}[]
\NormalTok{s <- (x - x_mean)^}\DecValTok{2}
\end{Highlighting}
\end{Shaded}
Modifique el archivo para que calcule \texttt{s} con este comando. Luego
vuelva a ejecutar \texttt{evaluar(1)} para ver la salida en la consola.
Debería encontrar las líneas ``\texttt{longitud de s ... OK}'' y
``\texttt{valores de s ... OK}'' al principio de la salida impresa.

\begin{center}\rule{3in}{0.4pt}\end{center}

\paragraph{Ejercicio (10):}

ahora sólo queda obtener el valor de \texttt{out} correcto (equivalente
a $\sigma ^ 2$ en la ecuación). Esta tarea queda para usted.

\begin{center}\rule{3in}{0.4pt}\end{center}

\paragraph{Recomendación general:}

muchas veces es buena idea usar un esquema, un dibujo, o cualquier
representación válida, antes de empezar a escribir código. Los problemas
de programación se pueden separar a groso modo en dos componentes: el
problema en sí y la forma de traducirlo a código. Por esto lo
recomendable es tratar de resolver el problema de base primero, en el
lenguaje que le resulte más cómodo. Esto puede ser en un diagrama de
flujo, un texto u otro medio. Posteriormente sólo quedará traducir su
solución al lenguaje R.

Esto no es un proceso lineal, ya que muchas veces una parte de su
esquema no tiene una traducción ``literal'' a código y por lo tanto
requiere de una revisión del enfoque. Es decir, se convierte en un
subproblema anidado en el problema principal. El esquema general que
hizo al principio será muy útil para que los subproblemas no se vuelvan
un obstáculo. Servirá de mapa para no perderse en un laberinto de
pequeños problemas.

Esta es una recomendación válida para los ejercicios del curso (que
esperamos sinceramente que no sean tan intrincados), pero es más válida
aún para la práctica de programar en general.

\subsubsection{3.4 Mensajes de advertencia}

Si hizo el ejercicio anterior, es posible que la corrección devuelva
mensajes de \emph{advertencia} además de mensajes de error. Veamos lo
que ocurre si usamos el siguiente código para calcular \texttt{out}:

\begin{Shaded}
\begin{Highlighting}[]
\NormalTok{out <- }\KeywordTok{sum}\NormalTok{(s)/}\KeywordTok{length}\NormalTok{(x) - }\DecValTok{1}
\end{Highlighting}
\end{Shaded}
Luego de guardar el archivo y ejecutar \texttt{evaluar(1)} se debería
imprimir estos mensajes de error y de advertencia:

\begin{verbatim}
* Al corregir el ej. 1, archivo 1-varianza.R:
Error : el valor de out obtenido no es el esperado,
la diferencia observada es la siguiente:

   Obs.  Esp.  Dif. 
  85.64 87.38  1.74

Warning message:
ej. 1: no es lo mismo 'a / (b + c)' que 'a / b + c' ...
\end{verbatim}
(Los valores exactos no van a ser los mismos, ya que siempre se usa un
nuevo \texttt{x} generado aleatoriamente.)

(``Warning'' es la palabra para advertencia en inglés.)

La intención de los dos tipos de mensajes es distinta. Los mensajes de
error son siempre objetivos; simplemente indican que no se resolvió
correctamente el problema. Usualmente se compara el valor esperado con
el obtenido.

Los mensajes de advertencia, cuando los hay, buscan dar una pista sobre
cuál puede ser el problema. Estos mensajes son fruto de la experiencia
que como profesores hemos tenido, la cual nos permite adelantarnos, a
grandes rasgos, a los errores de los estudiantes. No siempre habrá un
mensaje de advertencia oportuno para su error específico, ya que la
capacidad de predicción siempre es limitada. De todas formas, lo
importante es que usted sepa \textbf{aprovechar} estos mensajes, ya que
pueden ser de gran utilidad.

\begin{center}\rule{3in}{0.4pt}\end{center}

\paragraph{Ejercicio (11):}

resuelva por su propia cuenta el ejercicio 2 del repartido: ``Paradoja
de Zenón''. Tenga en cuenta que el ejercicio pide un \texttt{n}
específico, pero el resto del código debe ser genérico (en función del
\texttt{n} hallado).

\begin{center}\rule{3in}{0.4pt}\end{center}

\subsubsection{3.5 El puntaje y las notas}

Como ya vimos, al evaluar cualquier ejercicio con la función
\texttt{evaluar} se imprime en la consola un resumen de su performance
hasta ese momento. Puede también ver sus notas con la función
\texttt{verNotas}, la que imprime una tabla con un resumen de su
situación en el repartido. Las notas siempre se expresan en porcentaje:

\begin{verbatim}
> verNotas()
Parte     Nota Script        
1           1  1-varianza.R  
2           0  2-zenon.R     
3           1  3-extra-dist.R
Total (%) 100  --       
\end{verbatim}
Notará que en este caso se da un 100\%, a pesar de que el ejercicio 2 no
está completado. Esto se debe a que hay un ejercicio \emph{extra} ya
hecho. Los ejercicios extras son problemas más difíciles en general, que
sirven para que el estudiante interesado profundice algún aspecto del
curso, o simplemente para aumentar la nota total del repartido. Si usted
hace \emph{todos} los ejercicios, regulares y extras, entonces tendrá
una nota mayor a 100\%. Esta nota lo puede favorecer en el promedio de
repartidos, al final del curso. Sin embargo ese promedio no podrá
superar el 100\% (ver el \textbf{Apéndice II}).

Las notas se almacenan en el archivo ``datos'' que se encuentra en la
carpeta del repartido. Cada vez que se corre la función \texttt{evaluar}
dicho archivo se actualiza con sus notas (por esta razón, la carpeta del
repartido no puede estar bloqueada a escritura). Nótese que si usted no
usa la función \texttt{evaluar} para todos los ejercicios, aún si los
resolvió bien, su nota no lo reflejará.

Para entregar el repartido usted debe cargar en la página del curso el
archivo datos de su carpeta. Este será revisado posteriormente por los
profesores.

Este no es el único uso del archivo datos. Verá a continuación por qué
es central para el mecanismo de corrección.

\subsubsection{3.6 El archivo datos}

Además de las notas del estudiante, el archivo datos contiene las
funciones que usa \texttt{evaluar} para corregir los ejercicios de cada
repartido. Cada repartido tiene un archivo datos diferente.

Estas funciones siempre se están perfeccionando, debido a que se
descubren desperfectos o se mejoran ciertos aspectos, como el uso de
mensajes más informativos. Lamentablemente no siempre se pueden hacer
pruebas completas del funcionamiento de dichas funciones. Esto se debe a
que es muy difícil predecir todas las formas en que un estudiante puede
intentar resolver un ejercicio.

Por esto es importante mantener actualizado el archivo datos. Con este
objetivo se creó la función \texttt{fecha.datos}. Esta se carga en su
sesión cada vez que usted corre \texttt{source("evaluar.R")}. Veamos un
ejemplo:

\begin{verbatim}
> fecha.datos()
La fecha de su archivo datos es:
2013-08-28 13:34 CLT 
Link para ver fecha de la útima versión:
https://www.dropbox.com/s/3kwtwpa6mvq9lhn/fecha-datos-rep-1.txt 
Link para descargar la útima versión:
http://goo.gl/D5aYPW 
\end{verbatim}
Como puede ver, en la consola se imprime información relevante. Por un
lado, se indica la fecha de creación del archivo datos que usted tiene
en su carpeta. Por otro, se le da dos links: uno para verificar que la
fecha de su archivo coincide con la de la última versión disponible y
otro para descargar está última.

\begin{quote}
Nota: muchas veces en Windows, al descargar un nuevo archivo datos,
ocurre que el sistema le agrega la extensión .txt. Esto es un
comportamiento poco deseable, pero no es un problema. Cada vez que usted
corre \texttt{evaluar} R verifica si datos tiene alguna extensión y en
caso de que así sea, se borra (alternativamente, puede usar este comando
en R: \texttt{file.rename("datos.txt", "datos")}).

\end{quote}
\subsection{(4) Interfaz del foro}

Para el curso es muy importante la comunicación fluida a través de la
web. Para eso se utiliza un foro, alojado en
\href{reddit.com/r/imser}{reddit.com}. También puede enviar correos
electrónicos a \href{mailto:cursosr@gmail.com}{cursosr@gmail.com} o
simplemente conectarse por chat. De todas formas es mucho más provechoso
para todos plantear las dudas en el foro, ya que varios estudiantes
pueden compartir una duda y por lo tanto beneficiarse por la respuesta
recibida. Por esta razón, siempre se va a insistir en el uso por parte
de los profesores.

\subsubsection{4.1 Sobre reddit.com}

La web reddit.com es un sitio que sirve para crear comunidades online en
la que se comparten hipervínculos o textos. Las comunidades reciben el
nombre de ``subreddits'' o ``subs'' y tienen algún tema o consigna en
particular. Nuestro curso tiene el subreddit ``imser'', o como suele
referenciarse en estos casos, \href{reddit.com/r/imser}{r/imser}.

El único usuario moderador (por el momento) de este foro es u/imser y es
una cuenta manejada por los profesores del curso.

\begin{figure}[htbp]
\centering
\includegraphics{imagenes/reddit-suscripcion.png}
\caption{subscripción a r/imser.}
\end{figure}

\begin{center}\rule{3in}{0.4pt}\end{center}

\paragraph{Ejercicio (12):}

antes de empezar con este ejercicio, recomendamos continuar leyendo
hasta el final de este documento. Posteriormente: hágase un usuario
reddit y suscribase al r/imser. Luego comente en el post
\href{http://www.reddit.com/r/imser/comments/1la8s8/ejercicio\_final\_del\_repartido\_1\_suscribirse\_a/}{Ejercicio
final del repartido 1\ldots{}}, siguiendo las instrucciones que allí se
dan.

\begin{center}\rule{3in}{0.4pt}\end{center}

\subsubsection{4.2 Suscripción.}

Si usted ya hizo su usuario en reddit, ahora debe suscribirse al
r/imser. Para esto sólo tiene que ir a la url de r/imser
(reddit.com/r/imser) y apretar el botón verde a la derecha que dice
``suscribirse'' (tiene que haber iniciado sesión con su usuario, por
supuesto). En la figura 4 se muestra la interfaz de reddit y se indican
varios detalles importantes:

\begin{itemize}
\item
  La barra lateral, contiene información que usted debe leer; se trata
  de las reglas de uso del foro.
\item
  Los botones para crear nuevos posts. El de arriba es para crear un
  post a partir de un hipervínculo. El segundo es para crear un post de
  texto (cualquiera de los dos sirve y es posible hacer un post mixto).
\item
  La casilla de mail de reddit, indicada como ``PMs'', ya que PM es el
  acrónimo de ``Personal Messages'' (Mensajes Personales). El botón
  cambia de color a rojo cuando usted tiene nuevos mensajes.
\end{itemize}
\begin{figure}[htbp]
\centering
\includegraphics{imagenes/reddit-post.png}
\caption{creando un nuevo post en r/imser.}
\end{figure}

\begin{figure}[htbp]
\centering
\includegraphics{imagenes/reddit-post2.png}
\caption{leyendo un post}
\end{figure}

\subsubsection{4.3 Crear un nuevo post.}

Cuando usted quiera hacer un nuevo post, se encontrará con la interfaz
que ve en la figura 5. Dado que es una interfaz muy intuitiva, sólo
vamos a explicar aquí el estilo
\href{http://daringfireball.net/projects/markdown/syntax}{Markdown} para
dar formato al texto que escribimos en reddit.

El estilo Markdown es una forma de dar formato básico al texto que
resulta muy práctica y amigable (de hecho este mismo documento fue
creado usando Markdown). En particular, ya que este es un curso de
programación, nos interesa diferenciar con claridad lo que es
\texttt{código} de lo que es texto. El estilo convencional usa alguna
fuente ``\texttt{monospace}'' para esto. Para eso hay dos maneras
básicas:

\begin{enumerate}[1.]
\item
  Si el código está incluido dentro de una línea de texto cualquiera,
  usamos los acentos graves para delimitarlos. Por ejemplo:
  \texttt{`length(x)`} (en teclados en español suele encontrarse arriba
  del Shift derecho y a izquierda del Enter, o a la derecha de la Ñ).
\item
  Si el código debe estar en una o varias líneas aparte, hay que dejar 4
  espacios en blanco antes de cada línea. Por ejemplo:

\begin{verbatim}
Esto sería texto normal, antes de enviar el post...

    Esto sería código y se verá en alguna
    fuente monospace al enviar el post.
\end{verbatim}
\end{enumerate}
Cada vez que escriba un texto en reddit tome en cuenta estas y algunas
otras reglas de formato (no son muchas). Recuerde que tiene la
referencia en el borde mismo del espacio para escribir.

\subsubsection{4.4 Leyendo / guardando un post.}

Por último, en la figura 6 se muestra la interfaz de lectura de un post.
En este caso no hay comentarios aún. Puede ver que hay varias opiciones:

\begin{itemize}
\item
  Votar por el post. Cada usuario puede votar a favor o en contra de un
  post. Esto permite seleccionar por la cantidad de votos a los más
  útiles.
\item
  Guardar el post. Como usuario de reddit usted puede guardar posts que
  le resultan particularmente interesantes. Podrá acceder a ellos
  rápidamente en el futuro cuando usted lo desee.
\item
  Marcar como solucionado. Cuando la pregunta del OP (``Original
  Poster'', es el acrónimo estándar para referirse a quien inició la
  conversación) se ha respondido satisfactoriamente, usted puede
  agregarle la etiqueta de \emph{Solucionado}. Esto es muy útil para
  estudiantes y más aún para los profesores.
\end{itemize}
\begin{quote}
(Nota: la etiqueta solucionado es una versión modificada de la original
de reddit ``NSFW''; estando afuera de r/imser usted va a ver esa
etiqueta en lugar de ``Solucionado''.)

\end{quote}
\begin{center}\rule{3in}{0.4pt}\end{center}

\subsection{Apéndice I: instrucciones generales para los repartidos}

La información que aquí complementa la que se encuentra en el texto
principal.

\subsubsection{Archivos incluidos:}

El archivo con los ejercicios de cada práctico debe bajarse y
descomprimirse dentro de la carpeta del curso, creando la subcarpeta
\textbf{\texttt{rep-X}} (\texttt{X} es el número de repartido). Usted
deberá abrir RStudio y seleccionar dicha carpeta como su directorio de
trabajo con \texttt{setwd} o \textbf{Ctrl+Shift+K}. En esta carpeta se
encuentran algunos archivos que usted deberá modificar. En el archivo
\emph{letra.pdf} se indican cuáles son los archivos de ejercicio
(scripts de R) y cuáles son los archivos que usted no debe modificar.
Los ejercicios tienen nombres como \texttt{n-xxxxx.R}, en donde
\texttt{n} es el número de ejercicio.

\subsubsection{Mecanismo de corrección:}

Lo primero que debe hacer es cargar el archivo \texttt{evaluar.R} con la
función \texttt{source}:

\begin{Shaded}
\begin{Highlighting}[]
\KeywordTok{source}\NormalTok{(}\StringTok{"evaluar.R"}\NormalTok{, }\DataTypeTok{encoding =} \StringTok{"UTF-8"}\NormalTok{)}
\end{Highlighting}
\end{Shaded}
En caso de que ocurra un error o se vea otro mensaje en la consola,
verifique que los archivos se descomprimieron correctamente y que usted
está trabajando en la carpeta correspondiente con el comando
\texttt{getwd()}.

Cada vez que termina un ejercicio, o cuando los hizo todos, ejecute:

\begin{Shaded}
\begin{Highlighting}[]
\KeywordTok{evaluar}\NormalTok{()}
\end{Highlighting}
\end{Shaded}
Si usted no hace esto no habrá registro de sus notas en el archivo
``datos'' contenido en la carpeta del repartido. Recuerde que en todo
momento puede verificar su puntaje con la función \texttt{verNotas()}.

\subsubsection{Al finalizar}

Una vez terminados los ejercicios (y usada la función \texttt{evaluar}
para guardar sus notas), vaya a la página web del curso, a la sección de
entregas. Allí usted deberá subir el archivo \emph{datos}. Este será
corregido posteriormente por los profesores del curso.

\paragraph{Entregas con retraso:}

en el curso es posible entregar con retraso los ejercicios de los
repartidos, pero con un costo en el puntaje. Por ejemplo, si usted
entrega al día siguiente de la fecha límite, tendrá un 5\% de
amonestación. Cada día extra es otro 5\%: si en la página del curso se
indica que usted entregó con 4 días y 12 horas de retraso, entonces el
porcentaje de penalización total es de $5 \cdot 5 = 25$. Siguiendo el
ejemplo anterior, tendría una nota de 75\% para el repartido.

La fórmula general es, para D días y H horas de retraso (H
\textgreater{} 0):

\[
  P = (D + 1) \cdot 5
\]

(Cambia a $P = D \cdot 5$ si H = 0.)

\subsubsection{Código de Honor}

Si bien animamos a que trabaje en equipos y que haya un intercambio
fluido en los foros del curso, es fundamental que las respuestas a los
cuestionarios y ejercicios de programación sean fruto del trabajo
individual. En particular, no utilize el código creado por sus
compañeros, programe sus propias instrucciones. De lo contrario, estará
saboteando su propio proceso de aprendizaje. Esto también implica evitar
exponer el código propio a sus colegas. Como profesores estamos
comprometidos a dar las herramientas y explicaciones adecuadas a fin de
que pueda encontrar su propias soluciones para los ejercicios.

En casos de planteos de dudas a través del foro en los que considere que
es imposible no exponer su código, hágalo, pero recuerde que una vez
respondida la duda usted debe borrarlo. Si no lo hace los profesores
podrán borrar su pregunta original.

\begin{center}\rule{3in}{0.4pt}\end{center}

\subsection{Apéndice II: nota final del curso}

El siguiente código sirve para calcular la nota final a partir de dos
vectores: \texttt{notas.cuestionarios} y \texttt{notas.repartidos}.

\begin{Shaded}
\begin{Highlighting}[]
\NormalTok{nc <- }\KeywordTok{mean}\NormalTok{(notas.cuestionarios)}
\NormalTok{nr <- }\KeywordTok{min}\NormalTok{(}\DecValTok{100}\NormalTok{, }\KeywordTok{mean}\NormalTok{(notas.repartidos))  }\CommentTok{# El máximo posible es 100}
\NormalTok{nota.final <- }\DecValTok{100} \NormalTok{* (}\FloatTok{0.65} \NormalTok{* nr + }\FloatTok{0.35} \NormalTok{* nc)}
\end{Highlighting}
\end{Shaded}
\begin{center}\rule{3in}{0.4pt}\end{center}

\subsection{Soluciones de los ejercicios}

\subsubsection{1) Errores:}

\begin{enumerate}[1.]
\item
  En este caso hay un error en el nombre de la función: la función
  \texttt{mean} se escribe con minúsculas:

\begin{verbatim}
mean(5:7, na.rm = TRUE)
\end{verbatim}
\item
  Aquí hay dos interpretaciones posibles:
  \begin{enumerate}[a.]
  \item
    que falta una coma entre 8.564432 y el 3:

\begin{verbatim}
round(8.564432, 3)
\end{verbatim}
  \item
    que sobra el espacio en blanco entre 8.564432 y 3:

\begin{verbatim}
round(8.5644323)
\end{verbatim}
  \end{enumerate}
  No podemos resolver con mayor precisión el problema ya que no tenemos
  el contexto en el que se ejecuta el comando.
\item
  El objeto \texttt{bigcity} no se encuentra, debido a que el paquete
  boot no fue cargado en la sessión. Se puede comprobar que es un objeto
  perteneciente a este paquete con el comando:

\begin{verbatim}
??bigcity
\end{verbatim}
  Se soluciona el error cargando dicho paquete, así:

\begin{verbatim}
library(boot)
head(bigcity)
\end{verbatim}
\end{enumerate}
\begin{center}\rule{3in}{0.4pt}\end{center}

\subsubsection{2) Uso de la función mean:}

\begin{verbatim}
promedio <- mean(x)
\end{verbatim}
\begin{center}\rule{3in}{0.4pt}\end{center}

\subsubsection{3) Secuencias de números enteros:}

\begin{verbatim}
 10:10000
 20:10
 -8:6
 6:-8
\end{verbatim}
\begin{center}\rule{3in}{0.4pt}\end{center}

\subsubsection{4) Secuencias de números (seq):}

\begin{verbatim}
 seq(2, 110, by = 2)
 seq(1, 110, by = 2)
 seq(9, 0, length = 101)
\end{verbatim}
\begin{center}\rule{3in}{0.4pt}\end{center}

\subsubsection{5) Concatenación}

Ambas son aceptables:

\begin{verbatim}
mi.otro.vector <- c(45, -76, 3, 4, 5, 6, 0.333)
mi.otro.vector <- c(45, -76, 3:6, 0.333)
\end{verbatim}
\begin{center}\rule{3in}{0.4pt}\end{center}

\subsubsection{6) Modificación de un vector}

Ambas son aceptables:

\begin{verbatim}
mi.vector[2:3] <- c(100, 104)
mi.vector[c(2, 3)] <- c(100, 104)
\end{verbatim}
\begin{center}\rule{3in}{0.4pt}\end{center}

\subsubsection{7) Vector invertido}

\begin{verbatim}
mi.vector[6:1]
\end{verbatim}
\begin{center}\rule{3in}{0.4pt}\end{center}

\subsubsection{8) Matemática}

\begin{enumerate}[a.]
\item
  Para acceder a la ayuda (ambas sirven):

\begin{verbatim}
?log
help("log")
\end{verbatim}
  El argumento se llama ``base''

\begin{verbatim}
log(8, 3)
log(13, 5)
log(1.5, 10) # o log10(1.5)
log(7)       # o log(7, exp(1))
log(6, 2)    # o log2(6)
\end{verbatim}
\item
  La función se llama \texttt{range}. El argumento \texttt{finite} y
  puede tomar dos valores: TRUE o FALSE.

\begin{verbatim}
range(x)                # 1 y 100
range(x)                # -100 y -1
range(x, finite = TRUE) # 3.141593 y 2980.957987
\end{verbatim}
\item
  Polinomio:

\begin{verbatim}
x <- c(-1, 1.5, 4)
5 * x ^ 3 - 2 * x ^ 2 + 11
\end{verbatim}
\item
  Seno y coseno, con 100 valores de \texttt{x} aleatorios:

\begin{verbatim}
x <- runif(100, -1, 1)
cos(x) ** 2 + sin(x) ** 2
# o
cos(x) ^ 2 + sin(x) ^ 2
\end{verbatim}
\item
  Diagonal de un cuadrado:

\begin{verbatim}
a <- 1:100
D <- a * sqrt(2)
# o
D <- a * 2 ^ (1 / 2)
\end{verbatim}
\item
  Sumatoria (número $e$), para cualquiera de los valores de \texttt{n}:

\begin{verbatim}
i <- 0:n
sum(1 / factorial(i))
\end{verbatim}
\item
  Integral:

\begin{verbatim}
integral(log, 0, 20)
\end{verbatim}
\end{enumerate}
\begin{center}\rule{3in}{0.4pt}\end{center}

\subsubsection{9) Solución genérica}

\begin{verbatim}
mi.vector[length(mi.vector):1]
\end{verbatim}
\begin{center}\rule{3in}{0.4pt}\end{center}

\subsubsection{10) 1-varianza.R}

\begin{verbatim}
out <- sum(s) / (length(x) - 1)
\end{verbatim}
\begin{center}\rule{3in}{0.4pt}\end{center}

\subsubsection{11) 2-zenon.R:}

\begin{verbatim}
n <- 20
e <- 1:n
s <- 1 / (2 ** e)
out <- sum(s2)
\end{verbatim}
\begin{center}\rule{3in}{0.4pt}\end{center}

\subsubsection{Ejercicio extra del repartido}

\end{document}
