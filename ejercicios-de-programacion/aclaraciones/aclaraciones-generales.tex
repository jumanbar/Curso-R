\documentclass[]{article}
\usepackage{amssymb,amsmath}
\usepackage{ifxetex,ifluatex}
\ifxetex
  \usepackage{fontspec,xltxtra,xunicode}
  \defaultfontfeatures{Mapping=tex-text,Scale=MatchLowercase}
\else
  \ifluatex
    \usepackage{fontspec}
    \defaultfontfeatures{Mapping=tex-text,Scale=MatchLowercase}
  \else
    \usepackage[utf8]{inputenc}
  \fi
\fi
\usepackage{color}
\usepackage{fancyvrb}
\DefineShortVerb[commandchars=\\\{\}]{\|}
\DefineVerbatimEnvironment{Highlighting}{Verbatim}{commandchars=\\\{\}}
% Add ',fontsize=\small' for more characters per line
\newenvironment{Shaded}{}{}
\newcommand{\KeywordTok}[1]{\textcolor[rgb]{0.00,0.44,0.13}{\textbf{{#1}}}}
\newcommand{\DataTypeTok}[1]{\textcolor[rgb]{0.56,0.13,0.00}{{#1}}}
\newcommand{\DecValTok}[1]{\textcolor[rgb]{0.25,0.63,0.44}{{#1}}}
\newcommand{\BaseNTok}[1]{\textcolor[rgb]{0.25,0.63,0.44}{{#1}}}
\newcommand{\FloatTok}[1]{\textcolor[rgb]{0.25,0.63,0.44}{{#1}}}
\newcommand{\CharTok}[1]{\textcolor[rgb]{0.25,0.44,0.63}{{#1}}}
\newcommand{\StringTok}[1]{\textcolor[rgb]{0.25,0.44,0.63}{{#1}}}
\newcommand{\CommentTok}[1]{\textcolor[rgb]{0.38,0.63,0.69}{\textit{{#1}}}}
\newcommand{\OtherTok}[1]{\textcolor[rgb]{0.00,0.44,0.13}{{#1}}}
\newcommand{\AlertTok}[1]{\textcolor[rgb]{1.00,0.00,0.00}{\textbf{{#1}}}}
\newcommand{\FunctionTok}[1]{\textcolor[rgb]{0.02,0.16,0.49}{{#1}}}
\newcommand{\RegionMarkerTok}[1]{{#1}}
\newcommand{\ErrorTok}[1]{\textcolor[rgb]{1.00,0.00,0.00}{\textbf{{#1}}}}
\newcommand{\NormalTok}[1]{{#1}}
\ifxetex
  \usepackage[setpagesize=false, % page size defined by xetex
              unicode=false, % unicode breaks when used with xetex
              xetex,
              colorlinks=true,
              linkcolor=blue]{hyperref}
\else
  \usepackage[unicode=true,
              colorlinks=true,
              linkcolor=blue]{hyperref}
\fi
\hypersetup{breaklinks=true, pdfborder={0 0 0}}
\setlength{\parindent}{0pt}
\setlength{\parskip}{6pt plus 2pt minus 1pt}
\setlength{\emergencystretch}{3em}  % prevent overfull lines
\setcounter{secnumdepth}{0}


\begin{document}

\section{Dinámica de trabajo de los repartidos}

A lo largo del curso y en cada unidad, usted deberá completar repartidos
con los denominados \textbf{ejercicios de programación}. En estos se
pide al estudiante que complete un archivo de texto plano (un script de
R, y por lo tanto con la extensión \texttt{.R}) con el código (i.e.:
instrucciones en lenguaje R) necesarias para realizar las tareas
asignadas.

Para cada ejercicio usted podrá usar un sistema de corrección
automático, desarrollado para este curso, el cual le permitirá saber de
forma inmediata si ha completado correctamente las instrucciones del
mismo. Para que este método funcione correctamente, hay que seguir
ciertos procedimientos generales que pasaremos a describir a
continuación.

\subsection{Archivos incluidos:}

Cada repartido consta de una carpeta con archivos que usted puede bajar
desde la plataforma del curso (por ejemplo
\href{http://eva.universidad.edu.uy/file.php/1454/ejercicios\_de\_programacion/rep-1.zip}{este}).
Aquí por supuesto se encuentran los scripts de R que usted \textbf{debe
modificar}, así como ciertos archivos auxiliares que usted \textbf{no
debe modificar} (usualmente: \texttt{evaluar.R}, \texttt{datos},
\texttt{notas.csv} e \texttt{INSTRUCCIONES.pdf}, aunque pueden variar).
Estos archivos, particularmente los dos primeros, son esenciales para
que funcione el método de corrección automática que hemos desarrollado
para este curso.

La carpeta con el repartido debe bajarse y descomprimirse en disco duro,
creando la carpeta \textbf{\texttt{rep-X}} (siendo X el número de
repartido). Usted deberá abrir el RStudio y seleccionar dicha carpeta
como su directorio de trabajo con \texttt{setwd} (o en RStudio la
combinación \textbf{Ctrl + Shift + K})

\subsection{Escribiendo el código}

Cada uno de los scripts del repartido se corresponde con un ejercicio.
En cada archivo se indica con presición en dónde debe usted escribir su
código (o modificar lo que ya está escrito). \textbf{Cuide siempre de no
escribir ninguna línea de código ajena a los propósitos del ejercicio
mismo}. Por ejemplo, debe tener cuidado de no dejar escritas líneas como
\texttt{source(triangulo.R)} dentro del propio archivo
\texttt{triangulo.R}, ya que esto genera problemas al momento de evaluar
el ejercicio (básicamente es como pedirle al archivo que se evalúe a sí
mismo, lo que hace que se vuelva a evaluar a sí mismo y así sigue un
``circulo vicioso'', hasta que se llega a los límites de procesamiento
de la máquina). Por esta razón, recomendamos usar un script aparte para
este tipo de comandos, así como todos aquellos comandos usados para
experimentar con posibles soluciones o simples juegos de progamación que
usted quiera hacer. En el video ``Guía para repartidos'' mostramos un
ejemplo a seguir en caso de que esta explicación no le resulte
satisfactoria.

Nótese además que los cambios que se hacen al script del ejercicio son
\textbf{invisibles} para R hasta el momento en que usted \textbf{guarda}
el archivo a disco duro. Esta suele ser una fuente común de frustración
entre principiantes (y no tanto).

\subsection{Qué tipos de soluciones \emph{sirven}}

Por ejemplo, para el segundo ejercicio del repartido 1 (para el cual se
debe completar el código del script \texttt{areaMax.R}), no es válido el
siguiente comando:

\begin{Shaded}
\begin{Highlighting}[]
\NormalTok{i <- }\DecValTok{72}
\end{Highlighting}
\end{Shaded}
Si bien para el ejemplo que se ejecuta previo al ejercicio (i.e.: las
líneas de comando que aparecen en \texttt{areaMax.R} antes del código
que usted debe editar) esta solución es la correcta, \emph{no es una
solución} \emph{\textbf{general}} para el ejercicio. Es decir, si
cambiáramos el ejemplo, entonces 72 ya no sería una solución correcta,
ya que el valor máximo dentro del vector \texttt{a} seguramente se
encuentre ubicado en otra posición del mismo. El objetivo de estos
ejercicios, es que usted desarrolle soluciones \emph{universales} para
el problema en cuestión. Encontrar soluciones universales es en donde
radica el \textbf{poder de la programación} como herramienta, y por lo
tanto es nuestro objetivo en el curso.

Por esta razón, si usted prueba usar 72 como se muestra arriba, el
sistema de corrección automática lo dará como erróneo. En general para
cualquier ejercicio, el sistema de corrección automático utiliza números
aleatorios para verificar que su solución es robusta respecto a los
casos particulares (i.e.: diferentes valores numéricos).

\subsection{Objetos nombrados}

En varios ejercicios del curso usted se encontrarán con objetos en R que
contienen nombres. Por ejemplo, las columnas de una matriz de datos
(\texttt{data.frame} es el término correcto en R) tienen nombres. Parte
de la corrección automática consiste en verificar que estos nombres
estén correctos. Esto incluye: el \textbf{orden} en el que están los
nombres así como la \textbf{capitalización} de las letras en los nombres
(i.e.: mayúsculas y minúsculas). Por esto debe uster prestar atención a
estos detalles cada vez que realice una corrección automática de los
ejercicios.

\subsection{Mecanismo de corrección automática:}

Lo primero que debe hacer es cargar el archivo \texttt{evaluar.R} con la
función \texttt{source}, como se muestra a continuación:

\begin{Shaded}
\begin{Highlighting}[]
\KeywordTok{options}\NormalTok{(}\DataTypeTok{encoding =} \StringTok{"utf-8"}\NormalTok{)}
\KeywordTok{source}\NormalTok{(}\StringTok{"evaluar.R"}\NormalTok{)}
\end{Highlighting}
\end{Shaded}
Ejecutado correctamente, la siguiente frase debería verse en la consola:

\begin{verbatim}
Archivo de codigo fuente cargado correctamente
\end{verbatim}
En caso de que ocurra un error o se vea otro mensaje en la consola,
verifique que los archivos se descomprimieron correctamente y que usted
está trabajando en la carpeta correspondiente con el comando
\texttt{getwd()}.

Usted trabajará modificando los contenidos de dichos archivos con
RStudio (u otro programa de su preferencia) según las consignas que se
describen a continuación. Luego de terminar cada ejercicio y
\textbf{guardando el archivo} correspondiente en el disco duro, usted
podrá verificar rápidamente si su respuesta es correcta ejecutando el
comando:

\begin{Shaded}
\begin{Highlighting}[]
\KeywordTok{evaluar}\NormalTok{()}
\end{Highlighting}
\end{Shaded}
y además podrá en todo momento verificar su puntaje con la función
\texttt{verNotas()}. Una vez terminados los ejercicios del repartido,
usted deberá subir el archivo ''datos'' (sin extensión), incluido en la
carpeta ''rep-X'', a la
\href{http://eva.universidad.edu.uy/mod/assignment/view.php?id=93616}{sección
de entregas} de la portada del curso en la plataforma EVA.

\subsection{Sistema de puntaje}

Cada ejercicio en un repartido vale un punto, sin valores intermedios.
Es decir, el resultado es binario: 0 o 1 punto por ejercicio. Los
repartidos tienen ejercicios \textbf{obligatorios} pero también
ejercicios \emph{opcionales}. El puntaje total de cada repartido se
calcula como el porcentaje de puntos obtenidos respecto al total de
ejercicios \emph{obligatorios} del mismo. Por lo tanto, es posible
obtener notas mayores al 100\%. Por ejemplo: si hay 6 ejercicios
obligatorios y 2 opcionales, entonces 6 puntos equivalen a un 100\% y 8
puntos a 133\% ($nota = (puntos / 6) \cdot 100$).

Nótese que de todas formas, en la nota final del curso no se permitirán
porcentajes superiores al 100\%, en acordancia con el sistema de notas
de la UdelaR. Este sistema es simplemente una forma de motivar y poder
subir el promedio general de cada estudiante.

\subsection{Código de Honor}

Si bien animamos a que los estudiantes trabajen en equipos y que haya un
intercambio fluido en los foros del curso, es fundamental que las
respuestas a los cuestionarios y ejercicios de programación sean fruto
del trabajo individual. En particular, consideramos importante que los
estudiantes no miren el código creado por sus compañeros ya que esto
supone un sabotaje a su propio proceso de aprendizaje. Como profesores
estamos comprometidos a pedir tareas para las cuales hayamos dado las
herramientas correctas y las explicaciones adecuadas como para que todos
puedan encontrar su propio camino para resolver los ejercicios.

\end{document}
